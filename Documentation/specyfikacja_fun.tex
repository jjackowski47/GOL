\documentclass[12pt,a4paper,notitlepage]{report}

\usepackage{polski}

\usepackage[T1]{fontenc}


\usepackage[top=2cm, bottom=2cm, left=3cm, right=3cm]{geometry}
\usepackage{lastpage}
\usepackage{fancyhdr}
\pagestyle{fancy}
\fancyhf{}
\rfoot  { \centering \thepage \hspace{1pt} $\backslash$ \pageref{LastPage}}
\rhead{Automat komórkowy - Gra w życie}
\lhead{Specyfikacja funkcjonalna}

\makeatletter

\newcommand{\linia}{\rule{\linewidth}{0.4mm}}

\renewcommand{\maketitle}{\begin{titlepage}

\thispagestyle{fancy}

    \vspace*{1cm}

    \begin{center}
    
    \Large

    Specyfikacja funkcjonalna
    
    \end{center}

    \vspace{3cm}

    \noindent\linia

    \begin{center}

      \LARGE \textsc{\@title}

         \end{center}

     \noindent\linia

    \vspace{0.5cm}

    \begin{flushright}

    \begin{minipage}{5cm}

    \textit{\small Autorzy:}\\

    \normalsize \textsc{\@author} \par

    \end{minipage}


     \end{flushright}

    \vspace*{\stretch{6}}


  \end{titlepage}%

}

\makeatother

\author{Natalia Strawa\\
		Jakub Jackowski}

\title{Automat komórkowy - Gra w życie}
\renewcommand*\thesection{\arabic{section}} % zmiana numeracji sekcji 0.X -> X 
\begin{document}

\maketitle
\newpage
\tableofcontents
\setcounter{page}{2}
\thispagestyle{fancy}
\newpage

\section{Opis ogólny}
\subsection{Nazwa projektu}

Automat komórkowy - Gra w życie
\subsection{Nazwa programu}

\begin{verbatim}
gra_w_zycie
\end{verbatim}
\section{Opis ogólny}


\subsection{Poruszany problem}

Program będzie symulował automat komórkowy. Automat komórkowy to system składający się z pojedynczych komórek, znajdujących się obok siebie. Przylegające komórki tworzą siatkę dwuwymiarową. Każda z komórek może przyjąć jeden ze stanów - martwa/żywa. Stan wszystkich komórek w danej chwili to generacja.  Stan komórki zmieniany jest zgodnie z zasadami, które definiują, w jaki sposób nowy stan komórki zależy od jej obecnego stanu i stanu jej sąsiadów. Stany żywy i martwy będą oznaczane kolorami czarnym i białym. Program będzie umożliwiał utworzenie generacji, biorąc pod uwagę sąsiedztwo Moore'a: 
8 przylegających komórek (znajdujących się: na południu, na południowym-zachodzie, na zachodzie, na północnym-zachodzie, na północy, na północnym-wschodzie, na wschodzie i na południowym-wschodzie)
Zasady do utworzenia nowej generacji:
\begin{itemize}
	\item Martwa komórka, która ma dokładnie 3 żywych sąsiadów, staje się żywa.
	\item Żywa komórka z 2 albo 3 żywymi sąsiadami pozostaje nadal żywa; przy innej liczbie sąsiadów umiera.
	
\end{itemize}
\subsection{Użytkownik docelowy}

Studenci kierunków informatycznych zainteresowani tematem automatów komórkowych.
\section{Opis funkcjonalności}
\subsection{Jak korzystać z programu}
Aby uruchomić program, należy stworzyć plik z początkową generacją, wpisać wybrane flagi przy uruchomieniu i postępować według instrukcji znajdujących się na ekranie.
\subsection{Uruchomienie programu}
\begin{verbatim}
./gra_w_zycie -FILE file_path [ -FINAL file_path ] [ -STOP number_of_generations ] [ -PNG <a,b> | a b c … ] [ -TXT <a,b> | a b c … ] 
\end{verbatim}
albo
\begin{verbatim}
./gra_w_zycie -FILE file_path [ -SBS ]
\end{verbatim}
\subsection{Możliwości programu}
\begin{itemize}
\item Przeprowadzenie zadanej przez użytkownika liczby generacji.
\item Odczytywanie generacji z pliku.
\item Możliwość obserwacji stanu kolejnych generacji krok po kroku.
\item Generowanie obrazów przedstawiających stan wybranych generacji.
\item Zapisywanie bieżącej generacji do pliku.
\end{itemize}
\section{Format danych i struktura plików}
\subsection{Pojęcia i pola formularza (słownik)}
\textbf{Plansza} - zbiór przylegających do siebie komórek, w postaci dwuwymiarowej siatki.\\
\\
\textbf {Komórka} - pojedynczy element planszy, który może znajdować się w jednym z dwóch stanów, zgodnych z zasadami “Gry w życie” Johna Conwaya.\\
\\
\textbf {Generacja} - pojedyncza plansza przedstawiająca aktualny stan wszystkich komórek.\\
\\
\textbf {Martwa komórka} - komórka w stanie “martwym”, reprezentowana przez biały piksel w generacji.\\
\\
\textbf {Żywa komórka} - komórka w stanie “żywym”, reprezentowana przez czarny piksel w generacji.\\
\\
\textbf {Sąsiad} - komórka przylegająca do danej komórki, znajdująca się na: północy, północnym wschodzie, wschodzie, południowym wschodzie, południu, południowym zachodzie, zachodzie lub północnym zachodzie. Dla komórek, które znajdują się na brzegach siatki i nie mają niektórych sąsiadów, ustala się położenie sąsiada po drugiej stronie siatki od pierwotnie oczekiwanego sąsiada.
\subsection{Struktura katalogów}
Pliki Main.c, Board.c, Board.h, Life\_generator.c, Life\_generator.h, Make\_png.c, Make\_png.h oraz podkatalogi Tests, Images, Generations będą przechowywane w katalogu Gra\_w\_zycie.
\subsection{Przechowywanie danych w programie}
Generacje będą przechowywane w strukturze Board, która zawiera wskaźnik na tablicę dwuwymiarową charów, liczbę wierszy oraz liczbę kolumn.
\subsection{Dane wejściowe}
\begin{itemize}
	\item Plik z konfiguracją początkową.
\end{itemize}

\subsection{Dane wyjściowe}
\begin{itemize}
	\item Poszczególne generacje w postaci obrazów PNG.
	\item Poszczególne generacje w postaci plików tekstowych.
\end{itemize}

\section{Scenariusz działania programu}
\subsection{Scenariusz ogólny}
\begin{itemize}
\item Uruchomienie.
\item Wczytanie danych z pliku do struktury Board.
\item Przeprowadzenie zadanej przez użytkownika liczby generacji.
\item Zapisanie konfiguracji wynikowej do pliku i stworzenie na jego podstawie pliku PNG.
\item Zakończenie działania programu.
\end{itemize}
\subsection{Scenariusz szczegółowy}
\subsubsection{Uruchomienie bez opcji -SBS}
\begin{itemize}
\item Wczytanie danych z pliku i zapisanie do struktury Board,
w przypadku błędu związanego z otworzeniem pliku wyświetlany jest odpowiedni komunikat.
\item Wykonanie zadanej przez użytkownika liczby generacji,
w przypadku braku limitu generacje wykonane zostaną domyślną liczbę razy.
\item W przypadku podania flag -PNG lub -TXT odpowiednie generacje zostaną zapisane w formacie PNG lub do pliku tekstowego,
w przypadku podania numerów generacji wybiegających poza ich ostateczną liczbę zapisane zostaną jedynie generacje mieszczące się w ostatecznym zakresie.
\item Wygenerowanie końcowej generacji i zapisanie jej do pliku tekstowego oraz w formacie PNG.
\end{itemize}
\subsubsection{Uruchomienie z opcją -SBS}
\begin{itemize}
\item Wczytanie danych z pliku i zapisanie do struktury Board,
w przypadku błędu związanego z otworzeniem pliku wyświetlany jest odpowiedni komunikat.
\item Wydrukowanie na stdout poglądowej generacji.
\item Zapytanie użytkownika o wybór jednej z 4 opcji:
\begin{itemize}
\item Zapis aktualnej generacji do pliku tekstowego.
\item Zapis aktualnej generacji w formacie PNG.
\item Przejście do następnej generacji.
\item Zakończenie programu.
\end{itemize}
\item Zapisanie końcowej generacji do pliku tekstowego oraz w formacie PNG.
\end{itemize}
\section{Testowanie}
\subsection{Ogólny przebieg testowania}
Poszczególne funkcje w programie będą testowane za pomocą testów jednostkowych pisanych ręcznie podczas tworzenia programu.
\end{document}