\documentclass[12pt,a4paper,notitlepage]{report}

\usepackage{polski}

\usepackage[T1]{fontenc}


\usepackage[top=2cm, bottom=2cm, left=3cm, right=3cm]{geometry}
\usepackage{lastpage}
\usepackage{fancyhdr}
\pagestyle{fancy}
\fancyhf{}
\rfoot  { \centering \thepage \hspace{1pt} $\backslash$ \pageref{LastPage}}
\rhead{Automat komórkowy - Gra w życie}
\lhead{Specyfikacja implementacyjna}

\makeatletter

\newcommand{\linia}{\rule{\linewidth}{0.4mm}}

\renewcommand{\maketitle}{\begin{titlepage}

\thispagestyle{fancy}

    \vspace*{1cm}

    \begin{center}
    
    \Large

    Specyfikacja implementacyjna
    
    \end{center}

    \vspace{3cm}

    \noindent\linia

    \begin{center}

      \LARGE \textsc{\@title}

         \end{center}

     \noindent\linia

    \vspace{0.5cm}

    \begin{flushright}

    \begin{minipage}{5cm}

    \textit{\small Autorzy:}\\

    \normalsize \textsc{\@author} \par

    \end{minipage}


     \end{flushright}

    \vspace*{\stretch{6}}


  \end{titlepage}%

}

\makeatother

\author{Natalia Strawa\\
		Jakub Jackowski}

\title{Automat komórkowy - Gra w życie}
\renewcommand*\thesection{\arabic{section}} % zmiana numeracji sekcji 0.X -> X 
\begin{document}

\maketitle
\newpage
\tableofcontents
\setcounter{page}{2}
\thispagestyle{fancy}
\newpage

\section{Informacje ogólne}
\subsection{Uruchomienie programu z linii poleceń}
./gra\_w\_zycie -FILE file\_path [ -FINAL result\_file\_path ] [ -SBS ] [ -STOP number\_of\_ generations ] [ -PNG (a,b) | a b c … ] [ -TXT (a,b) | a b c … ] 

\subsection{Opis flag}
\begin{itemize}
	\item -FILE file\_path\\
	Użytkownik podaje ścieżkę do pliku z konfiguracją początkową, od której rozpocznie się symulacja “Gry w życie”.
	\item -FINAL file\_path\\
	Użytkownik podaje ścieżkę do pliku, w którym ma zostać zapisana ostatnia konfiguracja wygenerowana w danej symulacji “Gry w życie”.
	\item -SBS\\
	Pozwala uruchomić symulację w trybie “Step by step”. Użytkownik w każdym kroku otrzymuje aktualną generację, która zostaje zapisana w formacie png oraz zostaje zapytany o zapisanie aktualnej konfiguracji do pliku txt.
	\item -STOP number\_of\_generations\\
	Użytkownik podaje wartość graniczną ilości generacji, które mają się wykonać w danej symulacji.
	\item -PNG (a,b) | a b c ...\\
	Użytkownik podaje przedział generacji (a,b) od a do b włącznie lub wypisuje konkretne numery generacji a, b, c, … , z których chce utworzyć obrazy .png.
	\item -TXT (a,b) | a b c …\\
	Użytkownik podaje przedział generacji (a,b) od a do b włącznie lub wypisuje konkretne numery generacji a, b, c, … , które chce zapisać do pliku tekstowego
	
\end{itemize}	

\section{Opis modułów}

\subsection{Main.c}
\begin{itemize}
\item Wczytanie i parsowanie argumentów
\item Obsługa reszty modułów
\end{itemize}
Połączenia: File\_reader.c

\subsection{File\_reader.c}
\begin{itemize}
\item Czytanie pliku z konfiguracją
\item Zapisywanie konfiguracji do struktury Board
\end{itemize}
Połączenia: Main.c, Board.h

\subsection{Board.h}
Przechowywanie generacji:
\begin{itemize}
\item Liczba kolumn
\item Liczba wierszy
\item Macierz przedstawiająca daną generację
\end{itemize}
Połączenia: File\_reader.c, Life\_genrator.c

\subsection{Cell\_adjacency.c}
\begin{itemize}
\item Obliczenie ilości żywych sąsiadów dla danej komórki
\item Obsługa komórek leżących na obrzeżach planszy
\end{itemize}
Połączenia: Life\_genrator.c

\subsection{Life\_generator.c}
\begin{itemize}
\item Wytworzenie nowej generacji na podstawie reguł “Gry w życie” według Johna Conwaya
\item Zapisanie nowej generacji do tymczasowej planszy
	
\end{itemize}
Połączenia: Board.h, Cell\_adjacency.c, Save\_generation.c, Make\_png.c

\subsection{Save\_generation.c}
\begin{itemize}
\item Zapisanie generacji do pliku tekstowego
\end{itemize}
Połączenia: Life\_generator.c

\subsection{Make\_png.c}
\begin{itemize}
\item Stworzenie i zapisanie obrazu png przedstawiającego daną generację
\end{itemize}
Połączenia: Life\_generator.c


\section{Opis funkcji}

\subsection{Funkcja read}
\textbf {Prototyp}: Board* read(FILE* in)\\
\textbf {Zadanie}: Czytanie siatki z pliku tekstowego.\\
\textbf {Algorytm}:
\begin{enumerate} 
\item Funkcja czyta z pliku ilość wierszy i kolumn siatki oraz tworzy tablicę o takich wymiarach.
\item Funkcja wypełnia tablicę zerami.
\item Funkcja czyta z pliku współrzędne komórek tablicy, które przechowują wartość 1 oraz wypełnia te komórki jedynkami.
\end{enumerate}
\textbf {Wartość zwracana}: Wskaźnik na strukturę Board.

\subsection{Funkcja count\_neighbours}
\textbf {Prototyp}: int count\_neighbours (Board* board, int x, int y)\\
\textbf {Zadanie}: Liczenie ilość żywych sąsiadów dla danej komórki.\\
\textbf {Algorytm}:
\begin{enumerate} 
	\item Funkcja zlicza ilość żywych sąsiadów dla danej komórki o indeksach $\lbrack x \rbrack \lbrack y \rbrack$ sumując wartości komórek o indeksach od $\lbrack x+i\rbrack\lbrack y+j\rbrack$ gdzie $i,j \in C \wedge i,j \in \langle -1,1 \rangle$. Jeśli indeks $x+i$ jest mniejszy od zera lub równy ilości wierszy tabeli i tak samo jeśli indeks $y+j$ jest mniejszy od zera lub równy ilości kolumn tabeli, funkcja wywołuje funkcję edge oraz sumuje komórki o indeksach obliczonych przez funkcję edge.
	\item Funkcja odejmuje wartość komórki o indeksach $\lbrack x \rbrack \lbrack y \rbrack$ od obliczonej sumy.
	
\end{enumerate}
\textbf {Wartość zwracana}: liczba żywych sąsiadów danej komórki

\subsection{Funkcja print}
\textbf {Prototyp}: void print(Board* board)\\
\textbf {Zadanie}: Poglądowe wypisanie na stdout planszy z danej generacji\\
\textbf {Algorytm}:
\begin{enumerate} 
	\item Funkcja wypisuje górną krawędź planszy w postaci dwóch znaków minus “--” w ilości równej ilości kolumn planszy
	\item Funkcja iteruje po macierzy przechowującej daną generację i w przypadku 0 wypisuje dwa znaki spacji “  “, a w przypadku 1 dwuznak  “<>” symbolizujący żywą komórkę, na końcu każdego wiersza dopisując znak ”|” symbolizujący boczną krawędź planszy
	\item Funkcja wypisuje dolną krawędź planszy w postaci dwóch znaków minus “--” w ilości równej ilości kolumn planszy
	
	
\end{enumerate}
\textbf {Wartość zwracana}: brak

\subsection{Funkcja edge}
\textbf {Prototyp}: int edge(int x, unsigned int mod)\\
\textbf {Zadanie}: Ustalanie indeksów sąsiadów komórek znajdujących się na brzegach siatki.
\\
\textbf {Algorytm}:
Funkcja otrzymuje indeks x oraz liczbę mod (liczbę wierszy lub kolumn siatki), dodaje x do mod i liczy resztę z dzielenia x przez mod.\\
\textbf {Wartość zwracana}: indeks sąsiada danej komórki     

\subsection{Funkcja generate}
\textbf {Prototyp}: void generate (Board* generation)\\
\textbf {Zadanie}: Poglądowe wypisanie na stdout planszy z danej generacji\\
\textbf {Algorytm}:
\begin{enumerate} 
\item Funkcja tworzy tablicę o takich samych wymiarach jak siatka.
\item Funkcja iteruje po kolejnych komórkach tablicy (siatki), określa stan komórki w kolejnej generacji i zapisuje go do nowej tablicy:
\begin{itemize}
\item jeśli liczba sąsiadów komórki jest równa 3, komórka staje się żywa, 
\item jeśli liczba sąsiadów komórki jest różna od 2, komórka staje się martwa,
\item w innym wypadku stan komórki się nie zmienia.
\end{itemize}
	
\end{enumerate}
\textbf {Wartość zwracana}: brak


\section{Testowanie}
\subsection{Użyte narzędzia}

Pojedyncze funkcje w programie będą testowane przez napisane przez nas ręcznie testy jednostkowe. W miarę możliwości fragment programu napisany przez jedną osobę będzie testowany przez drugą, w celu wykrycia jak największej liczby błędów.

\subsection{Konwencja}
Pojedynczy test dla łatwiejszego zrozumienia jego przeznaczenia będzie nazywany w określony sposób:
nazwa\_testowanej\_funkcji().TEST.numer\_testu

Np. read().TEST.1

\subsection{Warunki brzegowe}
Szczególną uwagę podczas pisania testów zwrócimy na:
\begin{itemize}
	\item Jak program radzi sobie z gospodarowaniem pamięcią podczas zapisywania nowych generacji (czy nie ma nigdzie wycieków)
	\item Jak wydajny jest nasz program, czyli w jakim czasie jest w stanie wygenerować przykładowe duże ilości generacji
	\item Tworzenie obrazów png (czy obrazy są prawidłowo generowane, i czy są zgodne z odpowiednią generacją
	
\end{itemize}

\end{document}