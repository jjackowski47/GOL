\documentclass[12pt,a4paper,notitlepage]{report}

\usepackage{polski}

\usepackage[T1]{fontenc}
\usepackage[top=2cm, bottom=2cm, left=3cm, right=3cm]{geometry}
\usepackage{graphicx}
\usepackage{lastpage}
\usepackage{fancyhdr}

\pagestyle{fancy}
\fancyhf{}
\fancyfoot[C]{ \centering \thepage \hspace{1pt} $\backslash$ \pageref{LastPage}}
\rhead{Automat komórkowy - Gra w życie}
\lhead{Sprawozdanie końcowe}


\makeatletter

\newcommand{\linia}{\rule{\linewidth}{0.4mm}}

\renewcommand{\maketitle}{\begin{titlepage}
\thispagestyle{fancy}
		
		
		\vspace*{1cm}
		
		\begin{center}
			
			\Large
			
			Sprawozdanie końcowe
			
		\end{center}
		
		\vspace{3cm}
		
		\noindent\linia
		
		\begin{center}
			
			\LARGE \textsc{\@title}
			
		\end{center}
		
		\noindent\linia
		
		\vspace{0.5cm}
		
		\begin{flushright}
			
			\begin{minipage}{5cm}
				
				\textit{\small Autorzy:}\\
				
				\normalsize \textsc{\@author} \par
				
			\end{minipage}
			
			
		\end{flushright}
		
		\vspace*{\stretch{6}}
		
		
	\end{titlepage}%
	
}

\makeatother

\author{Natalia Strawa\\
	Jakub Jackowski}

\title{Automat komórkowy - Gra w życie}
\renewcommand*\thesection{\arabic{section}} % zmiana numeracji sekcji 0.X -> X 
\begin{document}

\maketitle
\newpage
\tableofcontents
\setcounter{page}{2}
\thispagestyle{fancy}
\newpage
\section{Specyfikacja funkcjonalna}

\subsection{Przeznaczenie}
\large\textbf{ Automat komórkowy - Gra w życie Johna Conwaya}\\
Program symuluje automat komórkowy “Gra w życie” według zasad Johna Conwaya. Z pliku zawierającego konfigurację początkową odczytuje wymiary planszy oraz współrzędne żywych komórek i na tej podstawie tworzy kolejne generacje. Dodatkowo program umożliwia zapis wybranych przez użytkownika generacji jako plik tekstowy lub obraz w formacie PNG. Program może działać w dwóch trybach “Step by step” - przedstawiając każdą generację w formie poglądowej i pytając użytkownika o wybór jednej z opcji oraz w trybie domyślnym z możliwością wyboru konkretnych generacji do zapisu.

\subsection{Wywołanie}
Program wywołujemy pisząc:
\begin{verbatim}
./gra_w_zycie [opcja] …
\end{verbatim}
gdzie dopuszczalne opcje to:
\begin{itemize}
	\item
	\begin{verbatim}
	 -FILE file_path
	\end{verbatim}
	Podaj ścieżkę do pliku z konfiguracją początkową, od której rozpocznie się symulacja “Gry w życie”.
	
	\item
	\begin{verbatim}
	-FINAL file_path
	\end{verbatim}
	Podaj ścieżkę do pliku, w którym ma zostać zapisana ostatnia konfiguracja wygenerowana w danej symulacji “Gry w życie”.
	
	\item
	\begin{verbatim}
	-SBS
	\end{verbatim}
	Uruchom symulację w trybie “step by step”. W każdym kroku jest możliwość wyboru jednej z czterech opcji: zapis aktualnej generacji w formacie PNG, zapis w formacie TXT, przejście do następnej generacji i zakończenie działania programu.
	
	\item
	\begin{verbatim}
	-STOP number_of_generations
	\end{verbatim}
	Podaj liczbę generacji, które mają się wykonać w danej symulacji.
	
	\item
	\begin{verbatim}
	-PNG <a,b> | a b c ...
	\end{verbatim}
	Podaj przedział generacji \verb|<a,b>| od \verb|a| do \verb|b| włącznie lub wypisz konkretne numery generacji \verb|a b c …| , z których chcesz utworzyć obrazy PNG.
	
	\item
	\begin{verbatim}
	-TXT <a,b> | a b c …
	\end{verbatim}
	Podaj przedział generacji \verb|<a,b>| od \verb|a| do \verb|b| włącznie lub wypisz konkretne numery generacji \verb|a b c …| , które chcesz zapisać do pliku tekstowego.
	
\end{itemize}

\subsection{Dane wejściowe}
Plik tekstowy o postaci
\begin{verbatim}
R C
i0 j0
i1 j1
i2 j2
…
\end{verbatim}
gdzie:
\verb|R,C| są liczbami naturalnymi oznaczającymi kolejno liczbę wierszy oraz kolumn planszy,
\verb|i0 j0, i1 j1, … , in jn|  są liczbami naturalnymi z zakresu \verb|<R, C>| oznaczającymi współrzędne komórek żywych.

\subsection{Format wyników}
Plik tekstowy o takiej samej strukturze jak dane wejściowe oraz plik w postaci obrazu w formacie PNG, na którym białe piksele oznaczają komórki żywe, a czarne - martwe.

\section{Architektura kodu}
Program składa się z czterech modułów: z modułu \verb|Main.c|, który wczytuje i parsuje argumenty oraz obsługuje resztę modułów, z modułu \verb|Board.h|, \verb|Board.c|, który służy do wczytywania, przechowywania oraz zapisywania generacji, z modułu \verb|Life_generator.h|, \verb|Life_generator.c|, który tworzy kolejne generacje komórek oraz z modułu \verb|Make_png.h|, \verb|Make_png.c|, który tworzy plik PNG przedstawiający daną generację i zapisuje go.

\newpage
\begin{flushleft}
\large\textbf{Bardziej szczegółowy opis modułów}
\end{flushleft}

\begin{enumerate} 
\item\textbf{Moduł} \verb|Main.c|\\

\textbf{Elementy modułu:}\\
\begin{itemize}
\item funkcja \verb|parse_range()| parsująca argumenty flag \verb|-TXT| oraz \verb|-PNG| w postaci zakresu \verb|"<a,b>"| i zapisująca granice do zmiennych \verb|from, to|

\item funkcja \verb|parse_list()| parsująca argumenty flag \verb|-TXT| oraz \verb|-PNG| w postaci listy indeksów generacji \verb|"a b c d …"| i zapisująca indeksy do wektora \verb|generation_index_list|

\item funkcja \verb|main()| obsługująca argumenty podane przy wywołaniu przy pomocy funkcji \verb|getopt()|, a następnie na ich podstawie uruchamia odpowiednie części programu lub wyświetla komunikaty o błędach

\end{itemize}

Połączenia: \verb|Board.c, Life_generator.c, Make_png.c|\\

\item\textbf{Moduł} \verb|Board.c Board.h|\\
\\
\textbf{Elementy modułu:}
\begin{itemize}
\item struktura \verb|Board| przechowująca generację: liczbę kolumn, liczbę wierszy oraz macierz przedstawiającą generację
\item funkcja \verb|read_config()| czytająca plik zawierający generację oraz zapisująca generację do struktury \verb|Board|
\item funkcja \verb|save_config()| zapisująca generację do pliku tekstowego
\end{itemize}
Połączenia: \verb|Main.c, Life_generator.c,  Make_png.c|\\

\item\textbf{Moduł} \verb|Life_generator.c Life_generator.h|\\
\\
\textbf{Elementy modułu:}
\begin{itemize}
\item funkcja \verb|generate()| tworząca nową generację na podstawie reguł “Gry w życie” według Johna Conwaya
\item funkcja \verb|count_neighbours()| obliczająca ilość żywych sąsiadów dla danej komórki
\item funkcja \verb|edge()| obliczająca indeksy sąsiadów dla danej komórki, gdy pierwotne indeksy wykraczają poza zakres planszy
\item funkcja \verb|print()| wypisująca na stdout planszę z daną generacją
\end{itemize}

Połączenia: \verb|Main.c, Board.c|

\item\textbf{Moduł} \verb|Make_png.c Make_png.h|\\
\\
\textbf{Elementy modułu:}
\begin{itemize}
\item funkcja \verb|process_file()| ustalająca kolor dla poszczególnych pikseli
\item funkcja \verb|write_png_file()| pisząca plik PNG na podstawie wyników utworzonych przez funkcję \verb|process_file()|.
\end{itemize}
Połączenia: \verb|Main.c, Board.c|\\
\\
\end{enumerate}
\begin{figure}[h!]
	\includegraphics[width=\linewidth]{diagram.jpg}
	\caption{Diagram modułów.}
\end{figure}

\section{Testy}
Pojedyncze funkcje w programie zostały przetestowane przez napisane przez nas ręcznie testy jednostkowe.
\begin{itemize}
	\item\textbf {Test funkcji} \verb|read_config|\\
	\\
	\textbf{Prototyp:} \verb|int does_read_config_read_correctly(FILE* in)|\\
	\\
	\textbf{Zadanie:} Sprawdzenie poprawności czytania pliku oraz zapisywania generacji do struktury \verb|Board| przez funkcję \verb|read_config|.\\
	\\
	\textbf{Algorytm:}
	\begin{enumerate}
	\item Funkcja \verb|read_config| czyta generację z pliku i zapisuje ją do struktury \verb|Board|.
	\item Funkcja testująca sprawdza, czy liczby kolumn i wierszy zgadzają się z zapisanymi w pliku, jeśli tak funkcja kontynuuje testy, w przeciwnym wypadku funkcja zwraca wartość \verb|0|.
	\item Funkcja testująca iteruje po tablicy przechowującej generację. Jeśli w komórce znajduje się \verb|1|, funkcja sprawdza, czy indeksy tej komórki zgadzają się z indeksami zapisanymi w pliku. Jeśli w komórce znajduje się wartość inna niż \verb|1| i \verb|0|, funkcja zwraca wartość \verb|0|.
	\item Jeśli funkcja testowana przeszła test pomyślnie, funkcja testująca zwraca wartość \verb|1|.
	\end{enumerate}

	Do testu został użyty plik o nazwie \verb|testowy_plik_z_konfiguracja.txt|.\\
	
	\item\textbf {Test funkcji} \verb|save_config|\\
	\\
	\textbf{Prototyp:} \verb|int does_save_config_save_correctly()|\\
	\\
	\textbf{Zadanie:} Sprawdzenie poprawności zapisywania generacji do pliku przez funkcję \verb|save_config|.\\
	\\
	\textbf{Algorytm:}
	\begin{enumerate}
		\item Funkcja \verb|read_config| czyta generację z pliku i zapisuje ją do struktury \verb|Board|.
		\item Funkcja \verb|save_config| zapisuje tą generację do pliku o nazwie \verb|save_config_test.txt|.
		\item Z racji tego, że korzystamy z tego samego pliku, co w teście funkcji \verb|read_config|, to do sprawdzenia poprawności zapisywania do pliku korzystamy z funkcji \verb|does_read_config_read_correctly|.
		\item Funkcja testująca zwraca wartość \verb|1| jeśli plik został zapisany poprawnie, a w przeciwnym wypadku wartość \verb|0|.
	\end{enumerate}

		Do testu zostały użyte pliki \verb|testowy_plik_z_konfiguracja.txt| oraz \verb|save_config_test.txt|.\\
		
	\item\textbf {Test funkcji} \verb|edge|\\
	\\
	\textbf{Prototyp:} \verb|int does_edge_work_correctly()|\\
	\\
	\textbf{Zadanie:} Sprawdzenie poprawności obliczeń wykonywanych przez funkcję \verb|edge|.\\
	\\
	\textbf{Algorytm:}
	\begin{enumerate}
	\item Funkcja testująca zwraca wartość \verb|0| jeśli funkcja edge dla przykładowych wywołań zwróciła nieprawidłową wartość.
	\item Jeśli funkcja testowana dla każdego wywołania zwróciła oczekiwaną wartość, funkcja testująca zwraca wartość \verb|1|.
	\end{enumerate}


	\item\textbf {Test funkcji} \verb| count_neighbours|\\
	\\
	\textbf{Prototyp:} \verb| int does_count_neighbours_count_correctly()|\\
	\\
	\textbf{Zadanie:} Sprawdzenie poprawności ustalania liczby sąsiadów dla danej komórki przez funkcję \verb|count_neighbours|.\\
	\\
	\textbf{Algorytm:}
	\begin{enumerate}
		\item Funkcja \verb|read_config| czyta generację z pliku i zapisuje ją do struktury \verb|Board|.
		\item Tablica \verb|index| przechowuje liczbę sąsiadów każdej komórki przykładowej generacji.
		\item Funkcja testująca iteruje po tablicy przechowującej generację i porównuje liczbę sąsiadów zwracaną przez funkcję \verb|count_neighbours| z wartością oczekiwaną z tablicy \verb|index|. Jeśli wartości się różnią funkcja zwraca wartość \verb|0|.
		\item Funkcja zwraca wartość \verb|1|, jeśli wszystkie wartości zwrócone przez funkcję testowaną są równe wartościom oczekiwanym.
	\end{enumerate}
Do testu zostały użyte pliki \verb|testowy_plik_z_konfiguracja_1.txt| oraz \verb|count_neighbours_test.txt|.
	
	\item\textbf {Test funkcji} \verb|generate|\\
	\\
	\textbf{Prototyp:} \verb|int does_generate_generate_correctly()|\\
	\\
	\textbf{Zadanie:} Sprawdzenie poprawności ustalania stanu komórki w następnej generacji przez funkcję \verb|generate|.\\
	\\
	\textbf{Algorytm:}
	\begin{enumerate}
		\item Funkcja \verb|read_config| czyta generację z pliku i zapisuje ją do struktury \verb|Board|.
		\item Funkcja \verb|generate| tworzy kolejną generację na podstawie wczytanej generacji.
		\item Funkcja \verb|read_config| czyta z pliku generację, która powinna być rezultatem działania funkcji generate.
		\item Funkcja testująca iteruje po tablicy przechowującej generację wygenerowaną przez funkcję \verb|generate| i porównuje ją z generacją oczekiwaną. Jeśli porównywane komórki różnią się od siebie funkcja zwraca wartość \verb|0|.
		\item Funkcja zwraca wartość \verb|1|, jeśli wartości wszystkich komórki generacji wygenerowanej przez funkcję \verb|generate| są równe wartościom oczekiwanym.
		\end{enumerate}

Do testu zostały użyte pliki \verb|testowy_plik_z_konfiguracja.txt| oraz\\ \verb|generate_test.txt|.\\

\item\textbf {Testy pamięci}\\
\\
Do sprawdzenia jak program radzi sobie z gospodarowaniem pamięcią użyliśmy narzędzia \verb|valgrind|. Przetestowaliśmy oba tryby:
\\
\\
\textbf{Bez opcji SBS}\\
\\
\small{
\verb|$ valgrind --leak-check=yes ./gra_w_zycie -FILE Configurations/flowers.txt|
\verb|-PNG "<10,41>" -TXT "<2,9>" -STOP 100 -FINAL abc|
\begin{verbatim}
==28719== Memcheck, a memory error detector
==28719== Copyright (C) 2002-2017, and GNU GPL'd, by Julian Seward et al.
==28719== Using Valgrind-3.13.0 and LibVEX; rerun with -h for copyright info
\end{verbatim}}
\verb|==28719== Command: ./gra_w_zycie -FILE Configurations/flowers.txt -PNG \<10,41\>|\\
\verb|-TXT \<2,9\> -STOP 100 -FINAL abc|
\begin{verbatim}
==28719==
==28719==
==28719== HEAP SUMMARY:
==28719==     in use at exit: 0 bytes in 0 blocks
==28719==   total heap usage: 6,587 allocs, 6,587 frees, 5,913,525 bytes allocated
==28719==
==28719== All heap blocks were freed -- no leaks are possible
==28719==
==28719== For counts of detected and suppressed errors, rerun with: -v
==28719== ERROR SUMMARY: 0 errors from 0 contexts (suppressed: 0 from 0)
\end{verbatim}

\large\textbf{z opcją SBS}
\small{
\begin{verbatim}
$ valgrind --leak-check=yes ./gra_w_zycie -FILE Configurations/flowers.txt -SBS

==28730== Memcheck, a memory error detector
==28730== Copyright (C) 2002-2017, and GNU GPL'd, by Julian Seward et al.
==28730== Using Valgrind-3.13.0 and LibVEX; rerun with -h for copyright info
==28730== Command: ./gra_w_zycie -FILE Configurations/flowers.txt -SBS
==28730==
==28730==
==28730== HEAP SUMMARY:
==28730==     in use at exit: 0 bytes in 0 blocks
==28730==   total heap usage: 483 allocs, 483 frees, 755,857 bytes allocated
==28730==
==28730== All heap blocks were freed -- no leaks are possible
==28730==
==28730== For counts of detected and suppressed errors, rerun with: -v
==28730== ERROR SUMMARY: 0 errors from 0 contexts (suppressed: 0 from 0)

\end{verbatim}}

\item\large\textbf {Testy wydajnościowe}\\
\\
Sprawdzamy ile czasu potrzebuje program aby przetworzyć różne wielkości danych wejściowych.
\begin{itemize}
	\item 10 generacji:
	
	\small\verb|$ time ./gra_w_zycie -FILE Configurations/flowers.txt -STOP 10 -PNG "<1,10>"|\\
	\verb|-TXT "<1,10>"|
	\begin{verbatim}
	real    0m0,039s
	user    0m0,003s
	sys     0m0,006s
	
	\end{verbatim}
	
	\large\item 100 generacji:
	
	\small\verb|$ time ./gra_w_zycie -FILE Configurations/flowers.txt -STOP 100 -PNG "<1,100>"|
	\verb|-TXT "<1,100>"|
	\begin{verbatim}
	real    0m0,357s
	user    0m0,037s
	sys     0m0,028s
	
	
	\end{verbatim}
	
	\large\item 1000 generacji:
	
	\small\verb|$ time ./gra_w_zycie -FILE Configurations/flowers.txt -STOP 1000|\\
	\verb|-PNG "<1,1000>" -TXT "<1,1000>"|
	\begin{verbatim}
	real    0m3,769s
	user    0m0,381s
	sys     0m0,264s
	
	\end{verbatim}
	
	\large\item 10000 generacji:
	
	\small\verb|$ time ./gra_w_zycie -FILE Configurations/flowers.txt -STOP 10000|\\
	\verb|-PNG "<1,10000>" -TXT "<1,10000>"|
	\begin{verbatim}
	real    0m37,972s
	user    0m3,976s
	sys     0m2,532s
	
	\end{verbatim}
	
	\large Duża część czasu działania programu jest wymuszona zapisem generacji, dla porównania test wydajności bez zapisu dla 10000 generacji:
	\small\begin{verbatim}
	$ time ./gra_w_zycie -FILE Configurations/flowers.txt -STOP 10000

	real    0m2,116s
	user    0m2,076s
	sys     0m0,020s
	\end{verbatim}
	
	
\end{itemize}
	
\end{itemize}

\section{Zmiany względem specyfikacji}
\subsection{Struktura katalogów}
Zostały utworzone nowe katalogi Configurations, w którym są przechowywane przykładowe konfiguracje początkowe, Dokumentacja, w którym są przechowywane wszystkie dokumenty związane z programem oraz plik Makefile z możliwością kompilacji programu, usuwania plików wygenerowanych podczas kompilacji oraz usuwania zawartości katalogów Images oraz Generations wygenerowanych podczas działania programu.

\subsection{Scenariusz działania programu}
Zrezygnowaliśmy z automatycznego zapisu końcowej generacji, uznaliśmy to za zbędną funkcjonalność, ponieważ służą do tego odpowiednie opcje wywołania programu.

\subsection{Nowe funkcje}
\textbf{Prototyp:} \verb|void parse_range(char *string, int *from, int *to)|\\
\\
\textbf{Zadanie:} Parsuje argumenty flag \verb|-TXT| oraz \verb|-PNG| w postaci zakresu \verb|"<a,b>"| i zapisuje graniczne wartości do zmiennych \verb|from, to|.
\begin{flushleft}
\textbf{Algorytm:}
Funkcja odczytuje argument w postaci \verb|"<a,b>"| i zapisuje wartość \verb|a| do zmiennej \verb|from| oraz \verb|b| do zmiennej \verb|to|.\\
\end{flushleft}
\begin{flushleft}
\textbf{Wartość zwracana:} brak
\end{flushleft}

\begin{flushleft}
\textbf{Prototyp:} \verb|int *parse_list(char *string)|\\
\end{flushleft}
\begin{flushleft}
\textbf{Zadanie:} Parsuje argumenty flag \verb|-TXT| oraz \verb|-PNG| w postaci listy indeksów generacji \verb|"a b c d …"| i zapisuje indeksy do wektora \verb|generation_index_list|.
\end{flushleft}
\begin{flushleft}
	\textbf{Algorytm:}
	Funkcja dzieli argument na oddzielne słowa, zlicza liczbę słów, a następnie każde słowo przekształca na liczbę całkowitą \verb|int| i zapisuje liczby do wektora.
\end{flushleft}
\begin{flushleft}
	\textbf{Wartość zwracana:} wskaźnik \verb|int*| do wektora zawierającego listę indeksów
\end{flushleft}
W funkcji zapisującej obrazy w formacie PNG została dodana nowa funkcjonalność zmiany rozdzielczości generowanych plików. Została utworzona zmienna \verb|resolution| o domyślnej wartości \verb|1|, która odpowiada za to jak bardzo powiększony ma zostać wygenerowany obraz.
\newpage
\section{Przykładowe wywołania}
\subsection{Wywołanie programu w trybie SBS:}
\begin{figure}[h!]
	\includegraphics[width=\linewidth]{wywolanie1.png}
	\caption{Wywołanie w trybie SBS.}
\end{figure}

\subsection{Wywołanie z błędną nazwą pliku, wyświetla instrukcję używania programu.}
\begin{figure}[h!]
	\includegraphics[width=\linewidth]{wywolanie2.png}
	\caption{Wywołanie z błędną nazwą pliku.}
\end{figure}
\newpage
\subsection{Przykładowe pliki wyjściowe}
\begin{figure}[h!]
	\includegraphics[width=\linewidth]{png.jpg}
	\caption{Pliki PNG dla całego przebiegu generacji z pliku flowers.txt (resolution = 10).}
\end{figure}
\begin{figure}[h!]
	\includegraphics[width=\linewidth]{szybowiec.jpg}
	\caption{Pliki PNG dla kilku kolejnych generacji z pliku glider\_gun.txt (resolution = 10).}
\end{figure}
\end{document}